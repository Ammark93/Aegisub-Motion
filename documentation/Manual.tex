\documentclass[a4paper,12pt]{article}
\usepackage{graphicx}
\usepackage{amsmath}
\usepackage[margin=2.5cm]{geometry}
\usepackage[linktoc=all]{hyperref}
\usepackage[all]{hypcap}
\usepackage{fancyhdr}
\usepackage{fontspec} % using xelatex like a professional
\usepackage{longtable}
\defaultfontfeatures{Ligatures=TeX}
\setsansfont[Scale=MatchLowercase]{Myriad Pro} % set sans font to Myriad Pro
\setmonofont[Scale=MatchLowercase]{Consolas} % set fixed-width front to Consolas
\setlength{\headheight}{15pt}
\pagestyle{fancy}
\fancyhead{}
\fancyfoot{}
\fancyhead[RO,RE]{Aegisub-Motion$\quad$\thepage}
\title{Aegisub-Motion Manual}
\author{}
\date{\today}
\renewcommand{\familydefault}{\sfdefault} % set default font to the default sans font
\renewcommand{\arraystretch}{1.3} % increase vertical spacing in tables
\begin{document}
  \maketitle
  \tableofcontents
  \newpage
  \section{Preamble}
  There have been a huge number of changes to the way this script operates since the last ``version'' was released. I have also decided to separate the sections on mocha usage into their own manual.
  
  While I'd like to believe that the usage of this script is pretty self explanatory, there are many things that have been added that cannot simply be discovered from the interface. The code in the script itself is definitely not laid out in the most understandable fashion, so I think it's only fair that a comprehensive explanation should be made. This manual is intended to cover all of the various features and functions of the script.

  \section{{\tt{}Motion Data - Apply}}
  \emph{But why do the progress bars go backwards?}

  \medskip

  Once you've gotten the motion data, run this. It is Aegisub-Motion's main interface and it allows you to define how the motion should be applied to the selected lines. It has tooltips for pretty much every option, so a few seconds of mouse hovering should be enough to generally learn what each does. The below table is a comprehensive explanation of what the options do---consider it supplemental material, though it may be useful in clarifying my poor descriptions.

  %labeled picture

  \begin{center}
    \begin{longtable}{rlp{11cm}}
      1 & Data & This is the tex box that the data is put in. It will accept either the path to a file or the raw data itself. By default, if the text on the clipboard contains the expected header, it will be copied in automatically when the interface runs. This feature can be disabled. \\
      2 & x \& y & These checkboxes specify which directions the position data should be applied. Their usage should be rather self explanatory: if only x is checked, then the resulting line will only move horizontally, and if only y is checked, the resulting line will only move vertically. This is useful in the case of perfectly horizontal or vertical motion that has acquired a bit of drift in the tracking process; however, under normal circumstances, both x and y should be checked. \\
      3 & Origin & The script does not handle perspective changes, but the goal of this is to preserve static {\tt\textbackslash{}frx} or {\tt\textbackslash{}fry} with a translating line by moving {\tt\textbackslash{}org} along with the position. Enabling it will automatically disable linear mode, as there is no way in ASS to translate the origin of a line. Note that it does not currently work properly if the line scales. This will hopefully be fixed in the future.\\
      4 & Clip & If this is checked, any existing {\tt\textbackslash{}(i)clip} in the selected lines will be transformed along with the line. At least x or y must also be checked, and all transformations applied to the line (scaling and rotation) will also be applied to the clip. The script does support moving the clip independently from the movement of the line, but that will be covered later, in the clip interface section. \\ % reintroduce higher resolution clipping?
      5 & Rounding & Specifies the number of decimal places to which all position data will be rounded. Two should be more than enough, but allowed values range from 0 to 5 decimal places. Note that this applies to the origin as well, but because clips must be defined by integer points, they will not be affected. \\
      6 & Scale & Defines whether or not the line should be scaled. By default, it will insert floating point scale values into the lines, which VSfilter does not work with. A VSfilter compatability mode exists, and you will probably want to use it, as many people still use VSfilter as their subtitle renderer. \\
      7 & Border & Defines whether or not border should be scaled with the line. Default is on. Will not apply if scale is not checked. \\
      8 & Shadow & Defines whether or not shadow should be scaled with the line. Default is on. Will not apply if scale is not checked. \\
      9 & Rounding & Specifies the number of decimal places to which scale, border, and shadow data should be rounded. Its valid range is 0 to 5, and the default is 2. \\
      10 & Rotation & Defines whether or not rotation data should be applied to the selected lines. Default is off. \\
      11 & Rounding & Specifies the number of decimal places to which rotation data should be rounded. Its valid range is 0 to 5, and the default is 2. \\
      12 & Write Config & If this is set, the current configuration will be written to the config file when the macro is run. This is useful for quickly changing options without having to run the config macro.\\
      13 & Start Frame & Aegisub-Motion supports applying tracking data from an arbitrary start frame within the frame range of the tracked data. Negative values count from the end so a value of -1 corresponds to the last frame, -2 corresponds to the second to last frame, and so on. Its valid range is from $-\mathrm{length}$ to $\mathrm{length}$, where $\mathrm{length}$ is the length of the selected set of lines in frames. 0 has the same result as 1---it corresponds to the first frame.\\
      14 & Relative & Aegisub-Motion supports tracking lines of different lengths simultaneously. This option defines whether each line's start frame should be relative to its own start (or end) time (checked) or if it should be relative to the start time of all selected lines (unchecked). Its default is on, as this replicates the old behavior (when the option was named ``reverse''). \\
      15 & VSfilter & VSfilter does not support floating point values to be set for {\tt\textbackslash{}fsc(xy)}. This option works around this issue by using staged transforms to represent intermediate scales, and is probably desired for normal usage. Its default is on---be warned that it increases the amount of computation required by the subtitle renderer, though.
    \end{longtable}
  \end{center}

  \section{Configuration}
  \emph{Aegisub-Motion attempts to read a configuration file every time one of its macros is executed. This config is used to override the hardcoded defaults on the user-end, for convenience. Aegisub-Motion can use per-project or global configuration files, which allows greater flexibility of use.}

  \medskip

  Aegisub-Motion uses a global variable at the top of the script (currently on line {\tt6}), called {\tt config\_file} to define the location of the config script. I \emph{strongly} recommend not changing the default, if only to save you the effort of having to change it every time you update the script. The comment at the top of the script explains how config loading works; I will rephrase it here, in case the meaning was unclear.
  
  If {\tt{}config\_file} is set to an absolute path (e.g. C:\textbackslash{}a-mo\textbackslash{}config.ext or /home/usernam/a-mo/config.ext), this will be used as a global config file, and the same file will always be loaded and written to.
  
  If {\tt{}config\_file} is set to a relative path, the behavior changes. Aegisub-Motion will attempt to load the config file from the same directory as the script. If this fails to load, it will fall back to the config file in Aegisub's userdata path (\%APPDATA\%\textbackslash{}Aegisub on Windows or \textasciitilde/.aegisub on a \*nix operating system). This means that you can have both a global config file (the one in userdata), as well as per-project configurations (assuming you keep all of the scripts from one project in the same directory). The usefulness of this will be explained in more detail with the rest of the config options.
  
  \subsection{Running{\tt{} Motion Data - Config}}
  \emph{This is a helper macro for generating your initial config and setting global options that cannot be set from the {\tt Motion Data - Apply} macro. The config file can also be edited manually, and its format is very simple to understand.}
  
  %screenshot etc

    
  \section*{Acknowledgements}
  I'd like to thank \textbackslash{}fake, \_\_ar, Hale, delamoo, nullx, zanget, Sutai, tophf and tp7 for their assistance, technical or otherwise, in writing this script. I'd also like to thank jfs and Plorkyeran (among others) for all the work they've done on Aegisub.  
\end{document}